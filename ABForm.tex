\documentclass[12pt]{report}
\usepackage{hyperref}
\hypersetup{
    colorlinks=true,
    linkcolor=blue,
    filecolor=magenta,
    urlcolor=blue,
    pdfpagemode=FullScreen,
}
\begin{document}

Preface: I am largely self taught in many of these areas and my education is largely confined to self guided reading with additional knowledge gained on the job. The books listed here I've mostly read through or still reference regularly. I also attempt to stay abreast of current trends and research by subscribing to \emph {Communications of the ACM} as well as taking advantage of being a member of the \href{https://www.acm.org/}{Association for Computing Machinery} which offers an extensive online publications library.\hfill \break

HARDWARE
\begin{itemize}
        \item \href{https://press.princeton.edu/books/ebook/9780691218960/understanding-the-digital-world}{\emph {Understanding the Digital World}} Kernighan, Brain
        \item \emph {Hacking the Xbox: An Introduction to Reverse Engineering} \href{https://en.wikipedia.org/wiki/Andrew_Huang_(hacker)}{Huang, Andrew}
\end{itemize}
I've been assembling computers since my early teens. I am well acquainted with peripherals (serial communication interfaces, GPIO, et al) and tinker with SoC devices regularly. I enjoy keeping older computers running (using OpenBSD) such as old Sun Ultras. I have a rudimentary understanding of how CPUs work and have some understanding of assembly language. I've also experimented with reverse engineering via ``hacking'' the original Xbox. I sometimes take apart devices to attempt to understand how they work. I have a basic understanding of electronic circuits.\hfill \break

SOFTWARE
\begin{itemize}
        \item \href{https://archive.org/details/designimplementa0000unse}{\emph {The Design and Implementation of the 4.3BSD UNIX\copyright Operating System}} Leffler, Samuel and McKusick, Marshall and Karels, Michael and Quarterman, John
        \item \href{https://en.wikipedia.org/wiki/The_Mythical_Man-Month}{\emph {The Mythical Man-Month}} Brooks, Frederick
        \item \href{https://en.wikipedia.org/wiki/TCP/IP_Illustrated}{\emph {TCP/IP Illustrated}} Stevens, Richard
        \item \href{https://en.wikipedia.org/wiki/The_Practice_of_Programming}{\emph {The Practice of Programming}} Kernighan, Brain and Pike, Rob
        \item \href{https://en.wikipedia.org/wiki/The_Unix_Programming_Environment}{\emph {The UNIX Programming Environment}} Kernighan, Brain and Pike, Rob - Learned much about Yacc (Yet Another Compiler-Compiler) and many other UNIX tools from this book
\end{itemize}
I attended a coding ``boot camp'' in the summer of 2015 which focused on Ruby on Rails and Node.js. This ``boot camp'' taught the basic tenets of software engineering and test driven development and introduced me to many different JavaScript frameworks and libraries such as Angular, React, Redux, and Express. I worked on a group project wherein we applied agile methods and produced a simple website for the Atlanta area community to contribute posts associated with the \href{https://en.wikipedia.org/wiki/BeltLine}{BeltLine}.

Since then I've been employed by various companies as a programmer and I've written and read programs in many different programming languages. I have a very good understanding of Git and version control. I've also worked with many different operating systems such as Linux, OpenBSD, and Windows.\hfill \break 

I earned a ``Master the Mainframe'' certificate from IBM in 2020 which demonstrates my programming skills and my ability to develop application tasks using C, Java, COBOL, assembler and REXX on the z/OS platform. I've also obtained a \href{https://ipv6.he.net/certification/}{Hurricane Electric IPv6 Certificate} in order to demonstrate my ability to configure IPv6.\hfill \break

THEORY AND ALGORITHMS
\begin{itemize}
        \item \href{https://en.wikipedia.org/wiki/Structure_and_Interpretation_of_Computer_Programs}{\emph {Structure and Interpretation of Computer Programs}} Abelson, Harold and Sussman, Gerald
        \item \href{https://www-cs-faculty.stanford.edu/~knuth/taocp.html}{\emph {The Art of Computer Programming Vol.1: Fundamental Algorithms}} Knuth, Donald
\end{itemize}
I have practical experience with data structures and algorithms and have implemented many of them in Python, Go, and JavaScript. I've also experimented  with a compiler written in C (Yacc mentioned above) and have a basic understanding of how compilers work. I've studied a wide variety of sorting algorithms such as quicksort, mergesort, and heapsort. I've also studied a variety of data structures such as linked lists, stacks, queues, trees, and graphs. I've also watched portions of CS50 from Harvard to garner some fundamentals.\hfill \break

APPLICATIONS
\begin{itemize}
  \item \emph {Designing Data-Intensive Applications} Kleppmann, Martin - I actually faced a problem at work highlighted in this book; ``clock skew'' and was able to diagnose and fix it.
\end{itemize}
I regularly work with PostgreSQL and MySQL and worked with NoSQL database systems in the past. I've experimented with TensorFlow and have a very limited understanding of machine learning. I've also worked with Docker and Kubernetes and have a good understanding of containerization and orchestration. I am a regular user of Nix and NixOS and contribute where I can to that ecosystem and community.\hfill \break

\end{document}
