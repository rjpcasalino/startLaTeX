\documentclass[12pt]{article}
\begin{document}
I am a dyslexic, Brooklyn born---Brooklyn, that somber city---and go at things as I have taught myself, free-style and will make the record in my own way while keeping it real and true.
Here I am thinking out loud and typing into a terminal. What I know is this: the cathode-ray tube of my youth has become a flat screen but it's all still the same. Nothing ends just begins again. What comes next for me? I've made it to being able to work with \LaTeX\ and I've taught myself how to read computer programs and scripts but something is lacking besides a Computer Science degree. Something I can't get from reading Knuth or Kernighan. Something I can't get from OpenBSD manual pages.

The areas I am most interested in include: practical applications for algorithms and the know how when it comes to selecting which one to apply for a given situation, systems programming to learn more about low-level programming, and hardware design to better understand how these computers actually work.\\

\hrule\
\\

My professional life starts in Chicago where I was designated an IT ``Power User'' at American Apparel for its stores in the region. I helped HQ in LA debug local point of sale system failures and replaced store servers when they went bust or needed software updates.

I moved away from Chicago to New Orleans where my IT experienced helped me land a job at a charter school managing its IT infrastructure and assisting teachers and students with all their computer and internet issues. I managed fleets of laptops and ensured internet access during critical times such as Statewide testing. Taking a step back, I had begun programming when I was ten years old but became convinced without evidence that beyond HTML all other languages required one to know binary which would require one to write ones and zeroes. During my last year at the charter school I discovered that this wasn't at all the case and programming was indeed a strong passion of mine and not all ones and zeroes, at least at the surface. I began to learn the rudiments of the Python programming language. These were challenging times for me as my job was ending and another major move was afoot because I had fallen in love. My future wife was moving to Atlanta to gain an MBA and she convinced me to better myself and apply to a coding boot camp. I did and I was able to gain employment as a programmer before the boot camp ended.

The thought of my first programming interview is terrifying. I had known that ``white boarding'' was in store but I was lucky enough to be interviewed by someone that wanted to pair program and tackle a problem together. I bombed it due to nerves but was again lucky my interviewer and future boss was kind and saw that I was, indeed, a programmer. He told me to go home and try again. I did and landed my first job as a ``Software Engineer''. It felt like a dream moving from one scene to the next in quick succession. I was able to work on a large code base and learn from my peers. I was able to write JavaScript based APIs and work on the front end. Soon I was worried I wasn't learning as much or as fast I should be and started seeking something new. My wife got a job in Seattle and we started packing our bags. 

I decided to apply to smaller startups to get a experience with that pace of working and it was a rush. I worked with talented people pushing hard to make an idea and company succeed and was given tasks that overwhelmed me at first but taught me much. I rewrote an entire back end (probably not the best idea) and the front end too. This experience taught me so much. Mostly what not to do. Anyway, the idea didn't catch hold and the startup failed but I don't regret my time there.

I worked at a few different companies thereafter both large and small before finally landing a job in the public sector. My current employer, Ad Hoc, is a federal contractor and I work on the Medicare.gov Plan Finder application. I've learned much in the last two years but it has also made me eager to further hone my craft and made me realize without a master's degree more doors would remain close and more interesting work/problems to solve would remain out of reach.

That is why I am applying to the PMP. Thank you for your time and consideration.
\end{document}
