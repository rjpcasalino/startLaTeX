\documentclass[12pt]{article}
\begin{document}
I am a dyslexic, Brooklyn born---Brooklyn, that somber city---and go at things as I have taught myself, free-style, and will make the record in my own way.
I began programming when I was ten years old but convinced myself that beyond HTML most other computer languages required one to know binary which would mean writing ones and zeroes. I nevertheless worked as a teenager on a website devoted to the Macintosh and its ecosystem and even worked on a blog written in PHP. I wrongly thought that \emph{real} programming was something beyond me. Soon the cathode-ray tube of my youth became a flat screen and I found myself in art school still writing HTML now and then and still knowing my way around computers despite my aversion to binary. Now I've made it to being able to work with \LaTeX\ and I've taught myself how to read and write computer programs, but something is lacking besides a Computer Science degree. Something I can't get from reading Knuth or Kernighan or OpenBSD manual pages alone.

The areas of study I am most interested in are: wireless sensor technologies such as Bluetooth and working with and better understanding such systems; practical applications for algorithms and how best to study patterns in code to recognize when certain algorithms have been employed for a given problem; human-computer interaction to better understand how user interfaces are designed and improved upon; systems programming to learn more about low-level programming and memory safety, and hardware design to better understand how computers fundamentally work and how hardware design decisions are made and why.

I believe the PMP can help me achieve a well rounded education in Computer Science beyond the fundamentals while also allowing me the opportunity to interact with other talented professionals to build a solid network for my future career growth. I believe I would be an asset to other students in the program because I very much enjoy helping others learn and have a track record of doing so in my professional career. I have a growth mindset which is essential in the area of Computer Science considering the rapid pace of change and innovation in the field.\\

\hrule\
\\

My professional life started in Chicago where I was employed as the IT ``Power User'' at American Apparel for its stores in the region. I helped HQ in Los Angeles debug local point of sale system failures and replaced store servers when they went bust or needed software updates.

I moved away from Chicago to New Orleans where my IT experience helped me land a job at a public Charter school managing its IT infrastructure and assisting teachers and students with all their computer and internet issues. I managed fleets of laptops and ensured internet access during critical times such as Statewide testing. As my time at the school was winding down I began to teach myself the rudiments of the Python programming language. These were challenging times for me as another major move was afoot because I had fallen in love. My future wife was moving to Atlanta to acquire an MBA and she convinced me to better myself and apply to a coding boot camp. I did and I was able to gain employment as a programmer before the boot camp ended.

Looking back, my first programming interview was terrifying. I had known of and done some ``white boarding'' but I was fortunate enough to be interviewed by someone that wanted to pair program rather than work a problem at the board. I let my nerves get the best of me nevertheless and did poorly but was again fortunate that my interviewer and future boss saw that I was, indeed, a programmer. He told me to go home, relax, and try the problem again alone. I did and landed my first job as a ``Software Engineer''. It felt like a dream moving from one scene to the next in quick succession. I worked on a RESTful JavaScript API as well as contributed to a front end application. Soon I was worried I wasn't learning as much as I should and started seeking something new.

My wife got a job in Seattle and I decided to apply to smaller startups to get experience with that pace of working. It was a rush. I worked with talented people pushing hard to make an idea and company succeed and was given tasks that overwhelmed me at first. I rewrote an entire back end (probably not the best idea) and the front end too. This experience taught me so much---mostly what not to do. Anyway, the idea didn't catch hold and the startup failed but I don't regret my time there.

I worked at a few different companies thereafter both large and small before landing a job in the public sector. My current employer, Ad Hoc, is a federal contractor and I work on the Medicare.gov Plan Finder application. I've learned much in the last two years but it has also made me eager to further hone my craft and made me realize without a master's degree more doors will remain closed and more interesting problems to solve will remain out of reach.

That is the main reason I am applying to the PMP. Thank you for your time and consideration.
\end{document}
