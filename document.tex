\documentclass{article}
\begin{document}
I am a dyslexic, Brooklyn born---Brooklyn, that somber city---and go at things as I have taught myself, free-style and will make the record in my own way. Thus begins my personal statement for why I am applying to the PMP.

Here I am thinking out loud and typing into a terminal. What I know is the cathode-ray tube of my youth has become a flat screen but it's all still the same. Nothing ends just begins again. What comes next for me? I've made it to being able to write with \LaTeX\ and I've taught myself how to read computer programs and scripts but something is lacking besides a Computer Science degree. Something I can't get from reading Knuth or Kernighan. Something I can't get from OpenBSD manual pages.

The areas I am most interested in include: practical applications for algorithms and how know which ones to apply in certain situations, systems programming to learn more about low-level programming, and hardware design to better understand how the machine actually works.\\

\hrule\
\\

My professional life starts in Chicago where I was designated an IT ``Power User'' at American Apparel for its stores in the region. I helped HQ in LA debug local point of sale system failures and replaced store servers when they went bust or needed software updates.

I moved away from Chicago to New Orleans where my IT experienced help me land a job at a charter school managing its IT infrastructure and helping its teachers and students with all their computer and internet issues. I managed fleets of Chromebooks and ensured internet access during critical times such as Statewide testing. Taking a step back, I had begun programming when I was ten years old but became convinced without evidence that beyond HTML all other languages required one to know binary and one wrote ones and zeroes. During my last year at the charter school I discovered that this wasn't so and programming was indeed my passion and not all ones and zeroes. I began to learn the rudiments of the Python programming language. These were challenging times for me as my job was ending and another major move was afoot because I had fallen in love. My future wife was moving to Atlanta to gain an MBA and she convinced me to better myself and apply to a coding boot camp in the area. I did and I was able to gain employment as a programmer before the boot camp ended.

The thought of my first programming interview was terrifying. I had known that ``white boarding'' was in store but I was lucky enough to be interviewed by someone that wanted to pair program and tackle a programming problem together. I bombed it nevertheless due to nerves but was again lucky my interviewer and future boss was kind and saw that I was, indeed, a programmer. He told me to go home and try again. I did and landed my first job as a ``Software Engineer''. It felt like a dream moving from one scene to the next in quick succession.

I was able to write JavaScript based APIs and work on front ends and it did all feel as if I were in a dream. Soon I was worried I wasn't learning as much or as fast I should be and got antsy and was seeking something else. My wife got a job in Seattle and we started packing our bags. I decided to apply to smaller startups to get a feel for that kind of pace and it was a rush. I worked with valeted people pushing hard to make an idea work and was given tasks that overwhelmed me but taught me much. I rewrote an entire back end (probably not the best idea) and the front end too. This experience taught me so much. Mostly what not to do. Anyway, the idea didn't catch hold and the startup failed but I don't regret my time there. 

I worked at a few different companies both bigger and smaller before I finally landing a job in a sector I had been seeking for some time. Ad Hoc is a federal contractor and I work on the Medicare.gov Plan Finder application. I've learned so much since being here but nevertheless after nearly two years it's made me eager to hone my craft and made me realize with a master's degree more doors would open and more interesting work/problems to solve would follow.
\end{document}
