\documentclass[12pt]{article}
\begin{document}
I am a dyslexic, Brooklyn born---Brooklyn, that somber city---and go at things as I have taught myself, free-style, and will make the record in my own way.
I began programming when I was ten years old but convinced myself that beyond HTML most other computer languages required one to know binary which would mean writing ones and zeroes. I nevertheless worked as a teenager on a website devoted to the Macintosh and its ecosystem and even worked on a blog written in PHP. I wrongly thought that \emph{real} programming was something beyond me. Soon the cathode-ray tube of my youth became a flat screen and I found myself in art school still writing HTML now and then and still knowing my way around computers despite my aversion to binary. Now I've made it to being able to work with \LaTeX\ and I've taught myself how to read and write computer programs, but something is lacking besides a Computer Science degree. Something I can't get from reading Knuth or Kernighan or OpenBSD manual pages alone.

I believe the Certificate in Computer Science Fundamentals can help me fill a void I've felt since becoming a professional programmer. Try as I might to fill it with the aforementioned books and manuals I can not shake the feeling that without guided instruction and a degree I'll never fill the void. Even though I attended a three month coding ``boot camp'' in 2015 and got a job before the program even ended I still feel the void. That is why I am applying to this program, to fill that void and hopefully move on to the Master of Science in Computer Science program. I believe I would be an asset to other students in this program because I very much enjoy helping people learn and have a track record of doing so in my professional career. I have a growth mindset which is essential in the area of Computer Science considering the rapid pace of change and innovation in the field. I've been working as a Software Programmer for nearly eight years using a wide variety of programming languages and tools but getting certificated that I understand and can work with and bandy about data structures, I/O, loops, Big O, and numerous other concepts would bring me a great sense of accomplishment and begin to fill that gnawing abyss in my heart.

My professional software programming journey began when I moved away from Chicago to New Orleans. My IT experience helped me secure a job at a public Charter school managing its IT infrastructure and assisting teachers and students with all their computer and internet issues. I managed fleets of laptops and ensured internet access during critical times such as Statewide testing. During my last year of employment there I began to teach myself the rudiments of the Python programming language. These were challenging times for me as another major move was afoot because I had fallen in love. My future wife was moving to Atlanta to acquire an MBA and she convinced me to apply to the coding ``boot camp'' previously mentioned.

My first programming interview was terrifying. I had done some ``white boarding'' but was fortunate enough to be interviewed by someone that wanted to pair program rather than work a problem at the board. I let my nerves get the best of me nevertheless and did poorly but was again fortunate that my interviewer and future boss saw that I was, indeed, a programmer. He told me to go home, relax, and try the problem again alone. I did and landed my first job as a ``Software Engineer''. It felt like a dream moving from one scene to the next in quick succession. I worked on a RESTful JavaScript API as well as contributed to a front end application. All this in an Agile environment with a team based in Finland. We'd converse about and debug the Angular front end and Node.js back end. The work was all about how best to present ``big data'' (HIPAA data, financial data, et al) in a digestible way with a UI that was easy to navigate and understand. I learned much at this first job from my managers and peers. I nevertheless felt that it couldn't be real, that I had lucked out and gotten a job that was too good to be true. I wanted to make sure I was challenging myself and I applied for another job writing JavaScript.

It was at my second job I was first exposed to a Java back end with a team working to build a front end using ``modern'' Node.js. Node.js was the rage then and understanding the event loop was a sure fire way to secure a position. But my time there was short as my wife got a job in Seattle and I decided to apply to smaller startups to get experience with that pace of working. And what a rush! I worked with talented people pushing hard to make an idea and company succeed and was given tasks that overwhelmed me at first. I rewrote an entire back end (probably not the best idea) and the front end too. The original MVC application was written with the Sails.js framework. Go was what people wanted so I set about rewriting core pieces in that language. This experience taught me so much---mostly what not to do. Anyway, the idea didn't catch hold and the startup failed but I don't regret my time there and my code had nothing to do with the business failure; product market fit et al but not the code. My code and the code of others worked well and handled thousands of requests a month. I am proud of that early code but I know so much more now that I wish I knew then.

Afterward, I worked at a few different companies both large and small before securing a job in the public sector. My current employer, Ad Hoc, is a federal contractor and I work on the Medicare.gov Plan Finder application. At Ad Hoc I've been working with gRPC and Go, JavaScript, and Ruby. I ship features to improve the experience for applying for Medicare and Medicaid. I also contribute to internal tools such as our ``People App'' which is a Ruby on Rails application for on boarding new staff and other associated tasks. I've learned much in the last two years but I've also realized not having a Bachelor's Degree in CS will only hinder my career growth. Having a Certificate in Computer Science Fundamentals will aid me on my path to a Master's Degree and future career opportunities and growth.

That is the main reason I am applying to Seattle University's Certificate in Computer Science Fundamentals program. They say with God all things are possible and I believe that's true. I also believe with the help of Seattle University's programs more things will be possible for me in the area of Computer Science, God willing. Thank you very much for your time and consideration.
\end{document}
